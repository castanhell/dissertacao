Para muitos problemas de otimização atualmente abordados encontrar a solução ótima é intratável , isto porque não conhecemos algoritmos que os solucionem em tempo polinomial. Sabendo desta limitação, podemos sacrificar a otimalidade (Por um valor conhecido) \cite{Vignatti} e aplicar uma solução tratável, em algo que chamamos de esquemas de aproximação polinomial. Veremos neste trabalho definições do que são estes esquemas de aproximação e alguns algoritmos criados a partir deste conceito para resolver quatro problemas: Problema da Mochila, Escalonamento de tarefas em máquinas paralelas, Empacotamento e Caixeiro viajante eucludiano. Apresentamos também a proposta de continuidade do trabalho, a qual definimos como sendo estudar algoritmos em grafos de disco unitários, em especial \cite{li}, onde recentemente descobriu--se um esquema de aproximação para o problema da cobertura de vértices em grafos de disco unitário com pesos.